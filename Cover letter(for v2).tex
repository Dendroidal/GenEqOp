% LaTeX file for resume 
% This file uses the resume document class (res.cls)

\documentclass{article} 
% the margin option causes section titles to appear to the left of body text 
\textwidth=5.2in % increase textwidth to get smaller right margin
%\usepackage{helvetica} % uses helvetica postscript font (download helvetica.sty)
%\usepackage{newcent}   % uses new century schoolbook postscript font 
\usepackage{url}
\usepackage{hyperref}
\hypersetup{
  % colorlinks,
  final,
  pdftitle={Equivariant Dendroidal Segal Spaces},
  pdfauthor={Bonventre, P. and Pereira, L. A.},
  % pdfsubject={Your subject here},
  % pdfkeywords={keyword1, keyword2},
  linktoc=page
}

\usepackage{xr}
\externaldocument{GenEqOp}

\input{commands.tex}%

%-------- TIKZ -----------------------------------------
\usepackage{tikz}%
\usetikzlibrary{matrix,arrows,decorations.pathmorphing,
cd,patterns,calc}
\tikzset{%
  treenode/.style = {shape=rectangle, rounded corners,%
                     draw, align=center,%
                     top color=white, bottom color=blue!20},%
  root/.style     = {treenode, font=\Large, bottom color=red!30},%
  env/.style      = {treenode, font=\ttfamily\normalsize},%
  dummy/.style    = {circle,draw,inner sep=0pt,minimum size=2mm}%
}%

\usetikzlibrary[decorations.pathreplacing]
% \usetikzlibrary{external}\tikzexternalize
% \makeatletters
% \renewcommand{\todo}[2][]{\tikzexternaldisable\@todo[#1]{#2}\tikzexternalenable}

% \makeatother

\begin{document} 
 
\title{Edits to ``Genuine Equivariant Operads'' (v.2)
\\[12pt]} % the \\[12pt] adds a blank line after name
 
\author{Bonventre, P. and Pereira, L. A.}
 
\maketitle
 
The referee's preliminary report had two main sections:
high level comments and in line comments.



\section{High level}


\section{Changes before report}

\begin{itemize}
\item Updated Gutierrez-White reference

\item Fixed last square diagram in the proof of Lemma 6.69

\item clarified diagram (3.1)

\item clarified 2 sentences after Remark 3.13

\item clarified diagram in Example 3.18

\item Defn 3.19 and Example 3.20 became Defn 3.23 and Example 3.24, while 
Defn 3.21, Remark 3.22, Prop 3.23 became Defn 3.19, Remark 3.20, Prop 3.21

\item Fixed some typos in Lemma 2.21
\end{itemize}


\section{Not changed}

\begin{itemize}

\item[34.] We are unsure how the notation in 
Remark \ref{LRROOTMAP REM} clashed with the notation in 
\eqref{OGDEF EQ} and Definition \ref{ROOTPULL DEF}.

\end{itemize}

\section{Direct response to comments}

\begin{itemize}
	\item[3.] We replaced ``iff'' with ``if and only if''
	in the expository/conversational parts of the paper, 
	but left it as ``iff'' in the technical parts (theorems, proofs, remarks), as we find it improves readability of many statements.
	
	Our previous experience is that ``iff'' seems to be an acceptable shorthand in publication, and that is our preference,
	but we do not have a strong objection to changing it.


        \item[4.] Replaced instances of $G \geq H \to \Sigma_n$ with less abbreviated notation throughout the paper.
        
      
	\item[7.] Added more detail concerning the 
	$H$-action on $N^X R$;
	moreover, introduced graph subgroups a little earlier in
	\eqref{GRAPHSUBIN EQ},
	and explicitly connected the action on 
	$N^X R$ to graph subgroups
	
	\item[8.] The meaning of the term ``family''
	was already spelled out the first time the term was used.

	\item[10.] we are not sure why the comment 
	``in order to work in the context of equivariant operads with norm maps (...) we need genuine equivariant operads''
	above \eqref{OPMULT EQ}
	would be interpreted as saying that we can't use equivariant operads obtain the correct homotopy theory.
	This sentence flowed directly from the last sentence of the previous paragraph, discussing why
	$\mathsf{Op}^{\mathsf{O}_G^{op}}$
	can't be used to obtain an Elmendorf-Piacenza style theorem if one wants $\mathsf{Op}^G$ to care about norm maps.
	Nonetheless, just to be safe, we've rephrased the comment to reiterate that we are discussing what is needed to obtain
	an Elmendorf-Piacenza theorem.

  \item[13.] Formally defined the $H$-space $\O(X)$ as the restriction of $\O(n)$ along the map $H \to \Gamma_X \leq G \times \Sigma_n$.
        
	\item[15.] Clarified references and definition of semi-model structures we require for this paper.
	The notion of semi-model structure we use was originally introduced in the unpublished \cite{Spi01} as a \textit{$J$-semi model structure},
	which agrees with the notion of a semi-model structure from \cite{WY18} for $\mathcal M = \**$.
	For non-trivial $\mathcal M$, the notions in \cite{Spi01} and \cite{WY18} do not agree;
	this is discussed further in the response to comment (71.).
	
	\item[16.] Concerning Theorem \ref{MAINEXIST2 THM},
	we are unsure as to why 
	having different $\mathcal{F}$ determine
	different categories makes our result
	``differ from the classical case''. 
	Partial genuine operads $\mathsf{Op}_{\mathcal{F}}(\mathcal{V})$
	work the same way as partial coefficient
	systems $\mathcal{V}^{\mathsf{O}_{\mathcal{F}}^{op}}$,
	with both being described as the subcategories
	of $\mathsf{Op}_{G}(\mathcal{V})$ and
	$\mathcal{V}^{\mathsf{O}_{G}^{op}}$
	that ``take the value $\emptyset$ away from $\mathcal{F}$''.
	Maybe we are thinking of different ``classical cases''? 
	
	\item[21.] we are unaware of any reference for Props. 
	\ref{FIBERKANMAP PROP} and \ref{GROTHSTAB PROP}.
	Certainly these results are elementary enough that they are probably already known (especially Prop. \ref{GROTHSTAB PROP}),
	but the results were new to us. 
	
	\item[24.] While these results are straightforward 
	and their proofs are generic, 
	we are unaware of any references for Propositions \ref{MONADADJ1 PROP}, \ref{MONADADJ PROP}, \ref{MONADICFUN PROP}. The likely references we are familiar with, such as 
	\cite{Bo94} or \cite{Ri17} do not quite discuss the framework we need.
	As with Props. 
	\ref{FIBERKANMAP PROP} and \ref{GROTHSTAB PROP},
	these results were new to us.
	
	\item[27.] In the intro to \S \ref{PLANAR_SECTION} we
	clarified that $\mathsf{F}^G$ denotes $G$-objects in $\mathsf{F}$,
	which is not a skeletal category (rather, it is the subcategory of $\mathsf{F}^G$ consisting of \emph{order preserving maps} that is skeletal).
	%
	This may seem pedantic,
	but it took us a fair amount of time to figure out the best level of ``skeletalness'' for the categories 
	$\Omega, \Phi, \mathsf{F}, \Omega_G, \cdots$, 
	which seems to be to demand that only planar/ordered maps are skeletal.
	
	\item[31.] the $\mathsf{O}^p_G$ notation in 
	Convention \ref{PLANARCONV CON}
	was a typo and should have been simply $\mathsf{O}_G$.
	The intended meaning is that we reinterpret totally ordered finite sets as having underlying set of the form
	$\{1<2<3<\cdots <n\}$,
	but this is already built into the definition of 
	$\mathsf{O}_G$ given in the introduction to
	\S \ref{PLANAR_SECTION}.

  \item[33.] Added a justification for 
	the characterization of ``independent maps in $\Omega_G$''
	provided in Remark \ref{INDOMGALT REM}.
	This required adding 
	Proposition \ref{INDMAPCHAR PROP},
	which provides an alternative characterization of \emph{independent maps}
	(than that in \cite[Def. 5.28]{Pe17})
	which is better suited for the discussion in this paper.
	Suitably altered the proof of Proposition \ref{PLANARPULL PROP}
	to refer to Proposition \ref{INDMAPCHAR PROP}
        
	\item[35.] more detail has been added to Example \ref{ROOTPULL EX}, hopefully making it easier to understand 
	how $\tau^{\**}(T)$ and $\pi^{\**}S$ are obtained.
	
	\item[36.] Example \ref{OUTERTREE EX} was added illustrating what outer faces look like pictorially. 
	Phrasing Definition \ref{OUTFACE DEF} 
	in terms of broad posets   
	is important to provide rigorous proofs
	
	\item[39.] Example \ref{PLANAREX EX} had been expanded provide examples of the $e^{\uparrow}$
	and $\underline{e} \leq e$
	notations; this should avoid the confusion concerning 
	Remark \ref{SCTARR REM}
	
	\item[46.] \eqref{PIIDEFDI2 EQ} does not need natural isomorphisms on the left, i.e. the square strictly commutes. The intuition is that since degeneracies merely duplicate one layer of vertex data, they do not change the way the vertex data is planarized. 
	
	\item[51.] 
	revised the proof of Lemma \ref{LANPULLCOMA LEM}
	to explicitly describe the retraction $r$ (instead of just describing the composite $r \circ V_G$)
	
	the comment preceding Corollary \label{MONDEFCOR COR}
	is indeed meant to refer to Lemma \ref{ROOTFIBPULL LEM}
	rather than Lemma \ref{LANPULLCOMA LEM}.
	The point is that Lemma \ref{ROOTFIBPULL LEM}
	is needed to state Corollary \label{MONDEFCOR COR}
	while 
	Lemma \ref{LANPULLCOMA LEM} is needed to prove it.

	\item[53.] Concerning Remark \ref{MUTMUT REM},
	and as pointed out by the referee,
	we do believe that all counterexamples require stumps,
	and thus that
	$\mathbb{F}_G \iota_{\**} \simeq 
	\iota_{\**} \iota^{\**} \mathbb{F}_G \iota_{\**}$
	holds in the context of reduced operads
	(on a side note, our argument does require $\V$ to satisfy most of the axioms in our main theorems, 
	so we set $\mathcal{V}=\mathsf{Set}$ in our counterexample
	to avoid an overstatement).
%	
	However, we're not sure it such a result is worth including in the paper.
	
	As an aside, the fact that 	
	$\mathbb{F}_G \iota_{\**} \simeq 
	\iota_{\**} \iota^{\**} \mathbb{F}_G \iota_{\**}$
	holds on cofibrant objects mostly comes down to
	Proposition \ref{FIXPT PROP} 
	and \eqref{COMCOFOB EQ}.
%	
	In order to prove the reduced operad analogue result, 
	Proposition \ref{FIXPT PROP} can be used as is, 
	but \eqref{COMCOFOB EQ} 
	(with $\mathcal{F}$-cofibrancy replaced with genuine cofibrancy)
	no longer follows from Corollary \ref{FINALCOR COR}
	and requires an entirely new proof that only works
	for trees with no stumps.
	
	\item[54.] concerning weak indexing systems,
	we find that Definition \ref{INDEXSYS DEF}
	is better suited for our purposes in this paper than 
	the definition in \cite{Pe17}; as such, even if we were to use the definition from \cite{Pe17} we would still need to convert it into a formulation better suited for this paper.
	
	On a side note, in the sequels to this paper 
	{\color{red} give references}
	we've been lead to identify further definitions,
	better suited for those papers.
	Personally, I (Luis) am prefer the Segal sieve definition
	
	
	\item[55.] throughout \S \ref{FREE_EXTENSIONS_SECTION}
	we replaced most of the abbreviations of
	$\mathbb{F}_G$ as $\mathbb{F}$ with the use of $\mathbb{F}_G$
	except for those in the long formulas
	\eqref{DOUBAR EQ} and \eqref{FURRES EQ},
	which wouldn't fit in the line and be hard to read otherwise
	(in both cases the abbreviation is mentioned immediately before the equation)

	\item[58.]
	we've spelled out Proposition \ref{PIIPROPAB PROP}
	a little more (this mostly came down to copying
	diagrams in Proposition \ref{PIIPROP PROP}
	with the symbols 
	$\Omega^n_G$
	replaced with 
	$\Omega^{n,s,\lambda}_{G}$ throughout)
	
	\item[59.] the proof of Corollary \ref{LABIDEN COR}
	was revised to add more detail and to make
	some arguments clear

	\item[60.]
	the confusing wording in the second paragraph of 
	Remark \ref{NPXY_REM} has been revised

	\item[61.] added Remark \ref{FILTCOMP REM},
	a fairly lengthy remark that compares our filtration argument with others in the literature, and referenced this remark near the beginning of \S \ref{FREE_EXTENSIONS_SECTION}. 

	As these filtrations are key to all our main proofs,
	there is \emph{a lot} that can be said on this topic.
	Remark \ref{FILTCOMP REM} focuses on the observations we felt
	were most useful for the reader,
	but here are some extra comments that were left out of 
	Remark \ref{FILTCOMP REM}.
	
	We followed neither of \cite{BM03},\cite{BB17} 
	when building our filtration \eqref{FILTR EQ}
	(though, of course, there are plenty of similarities).
	
	Instead, in the early stages of this work 
	we roughly followed the ``algebras over a colored operad'' framework of \cite{WY18},
	but even this comes with a caveat.
	The filtrations, and their proofs, in \cite{WY18}
	are a fairly direct colored analogue
	of work in \cite{Ha10},\cite{Ha09}.
	On the other hand, in \cite{Pe16}
	the second author provided a different approach to constructing the filtrations in \cite{Ha10},\cite{Ha09},
	which easily generalizes to the colored case.
	In concrete terms,
	this means that the motivation to construct 
	$\Omega^e_G$ comes from \cite{Pe16},
	since while \cite{WY18}
	features constructions that perform the role
	corresponding to 
	the $\widehat{\Omega}^e_G[k]$ 
	and $\widehat{\Omega}^e_G[k \setminus Y]$
	categories,
	it has no construction corresponding to the larger category $\Omega^e_G$. 
	
	On a related note, both 
	\cite{BB17} and \cite[App. C]{Cav}
	feature constructions that play the role of $\Omega^e_G$,
	though neither served as a reference for this work.



	\item[62.] 
	In order to better motivate the proof of 
	Proposition \ref{LXP PROP}
	we moved Example \ref{LRP EX}
	(which contains the pictorial description of $\mathsf{lr}_{\mathcal{P}}$) to before the proof,
	and added a brief reminder of the substitution setup in 
	Proposition \ref{BUILDABLE PROP}(iii)
	and Example \ref{GRAFTSUB EX}.

	\item[63.]
	We are not sure what was meant by the comment that
	``the picture in Example \ref{LRP EX} seems backward from
	Example \ref{REGALTERNMAP EX}''.
	Since $\mathsf{lr}_{\mathcal{P}}$
	is covariant, the two arrows should go in the same direction.
	Further, we expanded the informal description
	of $\mathsf{lr}_{\mathcal{P}}$
	in Example \ref{LRP EX},
	which should help clarify what this functor does.
	
	\item[64.]
	In Corollary \ref{KANRED COR} the sentence immediately after ``$\widehat{\Omega}^e_G \hookrightarrow \Omega^e_G$ is
	$\mathsf{Ran}$-initial over $\Sigma_G$''
	is the definition of $\mathsf{Ran}$-initial,
	so it seems the comment was already addressed.
	In addition, Lemma \ref{MINUS_LAN_FINAL_LEMMA} where ``$\mathsf{Ran}$-initial'' also appears was altered to refer to Corollary \ref{KANRED COR}.
	\item[82.] the proof of Prop. \ref{POWERF PROP} is terse since the overall argument is technically identical to that in the proof of [22, Thm 1.2] (one just modifies the hypothesis to get different conclusions).
	We would be happy to add extra detail if deemed necessary. 

	\item[66.] we are not sure what the intended issue with 
	``$lr(S)$ in 5.61'' 
	(during the proof of Lemma \ref{MINUS_LAN_FINAL_LEMMA})
	was meant to be.
	Nonetheless, we altered 
	Lemma \ref{MINUS_LAN_FINAL_LEMMA}
	to highlight that the forgetful functors to $\Sigma_G$
	are denoted $\mathsf{lr}$
	
	\item[67.] the short answer is that \eqref{EXTTREEFOR EQ} does define $\tilde{N}$.
	Ultimately, \eqref{EXTTREEFOR EQ} is just a slight restatement of Corollary \ref{ESTABDESC COR}, which builds 
	$\tilde{N}$ by applying Proposition \ref{RANTRANS PROP}.
	
	To avoid this confusion,
	we've added Remark \ref{TILNUNPACK REM} 
	which describes $\tilde{N}$ in fairly explicit terms
	(we had avoided including this since most of this explicit description isn't actually needed for our purposes,
	but maybe this will help clarify how 
	Proposition \ref{RANTRANS PROP} is being used),
	and revised the first few lines of 
	\S \ref{FILTRATION_SECTION} accordingly	

	
	\item[68.] As requested, additional detail was added to the proof of
	Proposition \ref{FILTINTALT PROP}

	\item[69.]
	To make the forward references less confusing, 
	we made sure that all forward references to results take place in
	Remark \ref{GEN_FGTRIGHT_REM},
	which was expanded accordingly.

	\item[71.] 
	Concerning the semi-model structure analogue of
	\cite[Thm. 11.3.2]{Hi03}
	(mentioned in the proof of Theorems \ref{MAINEXIST1 THM} and \ref{MAINEXIST2 THM}):
	
	Unfortunately, Spitzweck's notion
	of $J$-semi-model structure over $\mathcal M$ in \cite{Spi01}
	does not quite match the definition
	of semi-model structure
	we are using
	(which is based on more recent accounts e.g. \cite{Fre09,WY18}),
	so while \cite[Thm. 2]{Spi01}
	is indeed an analogue of \cite[Thm. 11.3.2]{Hi03},
	it features some extra technical conditions concerning cofibrancy.
	Namely, the \emph{definition} of the $J$-semi-model structure over $\mathcal M$ from \cite{Spi01}
	includes the assumption
	that transferred cofibrations between cofibrant objects forget to cofibrations.
	For example,
	in the context of Theorem \ref{MAINEXIST2 THM},
	this would mean that cofibrations between cofibrant objects in the \emph{projective} 
	model structure on $\mathsf{Op}_G$
	forget to cofibrations in the 
	\emph{projective} 
	model structure on $\mathsf{Sym}_G$.
	However, our argument instead shows that they forget to 
	cofibrations in the \emph{genuine} 
	model structure on $\mathsf{Sym}_G$
	(i.e. the model structure transferred from the left half of \eqref{MAINPFADJ EQ}).
	And since the projective model structure has more cofibrant objects but less weak equivalences, 
	this is sufficient to follow the argument in 
	\cite[Thm. 11.3.2]{Hi03} mutatis mutandis,
	which is what \cite[Thm. 2]{Spi01} does,
	\emph{except} \cite[Thm. 2]{Spi01}
	is using different cofibrancy assumptions than we are.

%	Spitzweck's notion of a $J$-semi-model structure from \cite{Spi01} includes as part of the \textit{definition}
%        that transfered cofibrations with cofibrant domain forget to cofibrations;
%        more recent accounts (e.g. \cite{Fre09,WY15}) prove this as a property of well-behaved semi-model categories,
%        as do we in our key Lemma \ref{MAINLEM LEM}).
%        After accounting for this difference (i.e. modifying the hypotheses to remove the condition on $\mathbb F I$-cells), the analogue result is \cite[Thm. 2]{Spi01}.
%        The relevant observation is that the proof of \cite[Thm. 2]{Spi01} is completely analogous to the proof of \cite[Thm. 11.3.2]{Hi03}.
        
	For clarity, we hence spelled out the ``natural analogue result'' of \cite[Thm. 11.3.2]{Hi03} we had in mind
	and added a footnote mentioning \cite[Thm. 2]{Spi01} as yet another variant of \cite[Thm. 11.3.2]{Hi03}.
	
	
	
        
	\item[72.] the argument in the proof of Theorem \ref{MAINEXIST2 THM}
	concerning the case with $\mathcal{F}$
	a general weak indexing system has been entirely rewritten and hopefully made clearer
	
	\item[73.] the paragraph explaining how the proof of 
	Theorem \ref{MAINEXIST1 THM}
	follows by adapting the proof of 
	Theorem \ref{MAINEXIST2 THM}
	has been expanded
	({\color{red} should add extra remarks somewhere about how the existence of the needed filtrations does not require $\V$ to have diagonals})

	\item[74.] Added short explanations of the double coset formulas used in 
	Propositions \ref{FGTLEFT PROP} and \ref{BIQUILLENG PROP}.

        \item[78.] added Remark \ref{CSPP_REM}, which compares the cofibrant symmetric pushout powers condition to other model categorical conditions of a similar form, and describe the utility of our definition.

        \item[79.] added Definition \ref{GENUINEMS_DEF} explicitly defining the genuine, $\F$, and projective model structures,
              and added language in \S \ref{PUSHPOW SEC} to emphasize which we are using.
	
	\item[81.] We are not sure what was meant by the 
	``where did the wreath product come from?'' comment
	concerning the proof of Proposition \ref{POWERF PROP},
	but we added further explanation concerning
	\eqref{GENCOFWR EQ} which might clear the confusion
	
	\item[84.] Further detail was added 
	to Remark \ref{UNPACKINGLTIMES REM} to better explain the given intersection
	
	\item[86.] we shrunk Remark \ref{WRONGSTRAT REM} 
	to focus on the comparison between Proposition \ref{AUTTCOFPUSH PROP} and \cite[Lemma 5.9]{BM08},
	and on which parts of our proof are similar and which are new,
	rather than go into detail as to why one can't simply adapt the proof of \cite[Lemma 5.9]{BM08} directly
	(the counter-example we built 
	is too complicated to be worth including,
	involving the group 
	$\Sigma_3 \wr \Sigma_2$ and 
	significant discussion of indexing systems)	
	
	\item[88.] the discussion concerning Proposition \ref{FIXPT PROP} has been significantly revised: 
	extra context concerning $\iota_{\**}$ was added beforehand; the proposition was restated to allow for an easier proof; and the proof was simplified and clarified

	\item[90.]
	we're not quite sure what is meant by the comment that
	the equation 
	$\iota^{\**} \mathbb{F}_G = \mathbb{F} \iota^{\**}$
	suggests one can go without $\mathbb{F}_G$.
	Since $\iota^{\**}$ simply forgets structure, the equation 
	$\iota^{\**} \mathbb{F}_G = \mathbb{F} \iota^{\**}$
	has very limited information, and is just the observation that the top levels of a genuine operad form a regular operad. 
	In particular, this 
	says nothing about norm maps.
	
	To make the relationship between 
	$\mathbb{F}_G $ and $\mathbb{F}$
	clearer, we've added to the discussion at the end of 
	\S \ref{COMPARISON_REGULAR_SECTION},
	namely discussing the
	easy identities
	$\iota^{\**} \mathbb{F}_G = \mathbb{F} \iota^{\**}$
	and
	$\mathbb{F}_G \iota_! = \iota_! \mathbb{F}$
	(neither of which is says anything substantial)
	in Remarks \ref{MUTIOTAUP REM} and \ref{MUTMUT1 REM},
	and the interesting functor
	$\mathbb{F}_G \iota_! \to \iota_! \mathbb{F}$
	in Remark \ref{MUTMUT2 REM}.
	

	\item[91.] the proof of Lemma \ref{MAINLEM LEM} has been revised at several points to clarify how the parallel argument connects
	\eqref{COFADJ2 EQ} and \eqref{FGTFUNC EQ}
	
	\item[92.] we're a little confused with regards to the comment that the proof Theorem \ref{MAINQUILLENEQUIV THM}
	would have needed to mention Kan complexes. As Kan complexes are simply the fibrant objects in $\mathsf{sSet}$,
	any necessary claims concerning Kan complexes
	are packaged instead in terms of fibrant objects;
	we've added an explanation as to why
	$\iota_{\**}$ detects fibrations
	
	\item[96.] Reworded and clarified the part of the proof of 
	Proposition \ref{BARCOF PROP}
	concerning the modified cube 
	$\tilde{\mathcal{X}}^n$
	and the associated appeal to 
	Lemma \ref{MONOCUBE LEM}(a).
	
	\item[99.] the superscript notation $(-)^{\downarrow j}$
	in Lemma \ref{UNDERLEFTADJ LEM}
	is not a typo, but is instead our chosen notation for the right adjoint to $(-)\downarrow j$.
	The proof as been expanded to include more detail.
\end{itemize} 

\section{Slight math fixes}

\begin{itemize}
	\item started fixing $G \times \Sigma_n$ 
	to 
	$G^{op} \times \Sigma_n$
	({\color{red} still incomplete})
	
	\item (hopefully) fixed the roles of $l$ and $m$ in 
	Proposition \ref{MONADICFUN PROP} and
	Remark \ref{PRECOMPPOSTCOMP REM}
	
	\item added Remark \ref{IOTAFUNSALT REM},
	which provides alternative formulas for
	$\iota_{\**}$
	
	\item said a little more at the end of the proof of
	Corollary \ref{IDEN COR}
	
	\item added Remark \ref{TOOPORNOT REM}, pointing out that it is preferable to set
	$\mathsf{Sym}^G(\mathcal{V}) = 
	\mathcal{V}^{G \times \Sigma^{op}}$
	(rather than $\mathcal{V}^{G \times \Sigma}$),
	and made changes throughout 
	\S \ref{COMPARISON_REGULAR_SECTION},
	\S \ref{MAINEXIST SEC},
	\S \ref{G_GRAPH_SECTION},
	\S \ref{MAINTHM_PROOF_SECTION}
	accordingly
	
	
	\item[28.] In the intro to \S \ref{PLANAR_SECTION}
	we replaced the reference to 
	\cite[Prop. 5.47]{Pe17}
	with a reference to 
	\cite[Def. 5.44]{Pe17},
	which in hindsight is indeed closer 
	to the formulation of the pullback \eqref{OGDEF EQ}.
	The connection to \cite[Prop. 5.47]{Pe17}
	does indeed require a little argument concerning Grothendieck constructions {\color{red} which can be found in one of our sequel papers}, 
	but we know of no canonical source.
	
	\item[47.] condition (c) in Proposition \ref{PIIPROP PROP}
	was fixed; the previous formulation was technically incorrect since the given permutations
	permute blocks in $V_G(T_k)$,
	and though the way the blocks
	are permuted depends only on 
	$T_0 \to T_1 \to \cdots T_i$,
	the size of the blocks also depends on the map 
	$T_i \to T_k$ 
	
	\item[85.] changed the $G \times \phi$ notation 
	to $id_G \times \phi$ in Remark \ref{LRLEFTQUILLEN REM}
	
	\item[89.] slightly modified the natural isomorphisms and added reference to Proposition \ref{MONAD_COMPARISON_PROP} 
\end{itemize}


\section{Minor and typos} 


\begin{itemize}
\item[1.] added accent to Guti\'{e}rrez's name throughout.

\item[4.] Replaced instances of $G \geq H \to \Sigma_n$ with less abbreviated notation throughout.

\item[5.] commas ({\color{red} will}) have been added 

\item[8.] clarified the definition of family

\item[9.] specified the reference to 
\cite{Elm83}
to refer to
\cite[\S 3]{Elm83}

\item[14.] replaced ``simplicial operad'' with  ``$G$-simplicial operad''
      
\item[17.] 
added the reference to the formulas for 
$\iota^{\**},\iota_{\**}$
in \eqref{IOTAFUNS EQ} 
before Theorem \ref{MAINQUILLENEQUIV THM}

\item[19.] added reference for the notion of ``Grothendieck fibration'' and reworded the definition of ``cartesian arrow''; fixed indicated grammar typo

\item[20.] fixed typos in Definition \ref{GROTHCONS DEF}

\item[21.] edited Prop. \ref{FIBERKANMAP PROP} to refer to the notation for a map of Grothendieck fibrations in \eqref{GROTHFIBMAP EQ}

\item[22.] fixed ``singleton''

\item[23.] fixed ``commute''. Wrong references were compilation error.
Fixed ``coproducts'' to ``monoidal product at the end of the proof''. Revised the $\downarrow_{\pi}$ notation to 
$\downarrow_{\mathsf{F}_s}$, 
to match the revised notation prior to 
Proposition \ref{FIBERKANMAP PROP}


\item[25.] Fixed typo. Added short description of the notion of ``module over a monad'' ({\color{red} add nLab reference? or find different one?})

\item[26.] the wrong reference seems to have been a compilation error

\item[29.] confirmed that the word ``predecessor'' (rather than ``sucessor'') is correct

\item[30.] added indication that, 
in Proposition \ref{PLANARIZATIONCHAR PROP},
$V(T)$ is the set of vertices

\item[32.] fixed the $\tilde{G}$ notation to $\bar{G}$ and reworded the example to clarify that $G,\bar{G}$ are two unrelated groups

\item[34.] revised the root functor $\mathsf{r}$ notation in pullback \eqref{OGDEF EQ} 
to match that in Definition \ref{ROOTPULL DEF}.
Following Definition \ref{ROOTPULL DEF},
added an indication to the category where 
$\varphi \colon Y \to X$ lives in.

\item[37.] added a line introducing the ``planar outer face map'' terminology. Provided references for the ``degeneracy-face decomposition'' 

\item[38.] no change was requested

\item[40.] added Notation \ref{STICKTRE NOT} introducing the ``stick tree'' terminology and referenced this notation before
Proposition \ref{BUILDABLE PROP},
clarifying the meaning of ``non-stick subtree''

\item[41.] the wrong reference seems to have been a compilation error

\item[42.] the wrong reference seems to have been a compilation error

\item[43.] the wrong reference seems to have been a compilation error

\item[44.] added Notation \ref{DDDDD NOT}, 
which defines $d_{1,\cdots,n}$ style notation; 
In the proof of Proposition \ref{SUBSASPULL PROP}
fixed ``injectivity'' to be ``surjectivity''; fixed ``after converted'' to ``once converted''

\item[45.] rewrote Notation \ref{INDVNG NOT}
to clarify the meaning of $\sigma^0$ 

\item[48.] replaced ``labelled'' with ``labeled'' throughout

\item[49.] further explained how groupoids lead to \eqref{FGXDEFEXP EQ}

\item[50.] clarified references to \cite[X.3.1]{McL} as \cite[X.3 Thm. 1]{McL}, which should make the ``pointwise formula for Ran'' easier to identify, and further referenced uses of this formula to \eqref{FIBERKAN EQ}.
Fixed one reference to \cite[X.3.1]{McL}
to be a reference to \cite[IX.3]{McL}.

\item[51.] fixed ``functors'' to ``natural transformations'' in Corollary \ref{MONDEFCOR COR}

\item[55.] added a footnote describing the double bar construction;
fixed a math typo where $\iota^{\times \langle l \rangle}$ should be 
$\iota^{\times \langle \langle l \rangle \rangle}$

\item[56.] 
Notations \ref{UEUPE NOT} and \ref{UEUPEG NOT}
were added to help clarify the notations
$U_{e^{\uparrow} \leq e}$
and
$S_{v_{Ge}}$
when $e$ comes from a different tree $T$.

The notation 
$U^{\mathsf{r}}_{v_{Ge}}$
previously used at the end of \S \ref{LRVERT SEC}
was replaced with just
$S_{v_{Ge}}$
throughout. Here the $U$ became $S$ in the equivariant context for the sake of consistency,
and the $\mathsf{r}$ was dropped since it was deemed unnecessary.
Initially we intended 
$U_{v_{Ge}} \to U$
to refer to the planar version of the map 
$U^{\mathsf{r}}_{v_{Ge}} \to U$,
but the former notion is never actually needed in practice.
Remark \ref{WHYALT REM} discusses this subtlety.

In Definition \ref{LABMAP DEF} we added a reference to
Notation \ref{UEUPEG NOT},
where $T_{v_{Ge}}$ is introduced;
we likewise added this notation to the notational index


\item[57.] fixed ``forgetful''

\item[62.] fixed ``passive'' to be ``inert''

\item[65.] noted that the use of opposite categories turns $\mathsf{Ran}$ into $\mathsf{Lan}$

\item[70.] replaced ``less generating'' with ``fewer generating''
      
\item[75.] Made sure the statement of Lemma \ref{REWORFAM LEM} makes sense

\item[76.] addressed in the answer to [3.]

\item[77.] fixed $G$ to be $\bar{G}$
      
\item[80.] reworded ``for all'' as ``for every'' in Prop. \ref{POWERF PROP}

\item[82.] rephrased ``retracts can be ignored''
as ``retracts preserve weak equivalences''

\item[83.] fixed ``trully'' to ``truly''

\item[87.] clarified which step requires cofibrancy of the 
$f_s$

\item[93.] added accent to Gutierrez's name throughout.

\item[94.] added \cite{Ri14} reference to the ``extra degeneracy argument''; fixed ``monormorphism'' typo

\item[95.] introduced the 
$\underline{n} = \{1,2,\cdots,n\}$
notation a little earlier in \S \ref{NINFTY_SECTION}

\item[97.] fixed wrong way quotation mark

\item[98.] rephrased the proof of Lemma \ref{OBJGENREL LEMMA}
to avoid $\in$ symbols in opposite directions

\item[100.] clarified the reference to Prop. \ref{FIBERKANMAP PROP} by specifying the relevant undercategories

\item[101.] updated bibliography.

\end{itemize}




\section{On free extensions}

the connection between our extension category and the approaches 
of Berger-Moerdijk in \cite{BM03}
and of Batanin-Berger in \cite{BB17}
is fairly convoluted.
From a ``big picture'' point of view all approaches 
are following the same basic intuition. 

However, the exact procedure we follow to identify 
the extension category seems to be new,
owing in part to the setup in 
Theorems \ref{MAINEXIST1 THM} and \ref{MAINEXIST2 THM},
which requires us to worry about things that 
\cite{BM03} and \cite{BB17} need not worry about
(more detail below).
To properly compare the approaches, we summarize them below:
\begin{itemize}
	\item[(LT)] the approach in 
	\cite[\S 5.11]{BM03}
	directly describes filtrations of free extensions as in
	\eqref{FILTRATION_LAN_SQUARE_DIAGRAM}
	where the left map is given as a colimit over labeled trees.
	Somewhat notably, the original description of 
	the filtration in \cite[\S 5.11]{BM03}
	was not correct due to ignoring the effect of the operadic unit,
	leading to a correction of that filtration in 
	\cite{BM09}.
	Also of relevance for this approach is Caviglia's work in Appendices B,C of \cite{Cav},
	which adapts the \cite{BM03} filtration 
	to the colored case, while also providing a more detailed discussion.
	\item[(PM)] 
	for polynomial monads $T$, \cite{BB17} describes free extensions as colimits over
	an ``internal classifier 
	$\bold{T}^{T_{f,g}}$ for free algebras''
	(the specific result is \cite[Thm. 6.17]{BB17}
	for $S = T_{f,g}$, where 
	$T_{f,g}$ is built in \cite[\S 7]{BB17}).
	For \emph{tame} polynomial monads, 
	the desired filtrations of free extensions are then 
	built by identifying a final subcategory of
	$\bold{T}^{T_{f,g}}$,
	and filtering that subcategory.
	\item[(CO)]
	\cite[Prop. 4.3.16]{WY18} describes a filtration 
	of free extensions of algebras over a 
	\emph{colored} symmetric operad. 
	The relevance of this is given by
	the fact that there exists a colored operad
	$\mathsf{Op}$ whose algebras are the classical single colored operads
	(informally, colors in $\mathsf{Op}$
	are the set of arities $n \geq 0$
	and operations are trees, with arities of vertices being the input colors and number of leaves being the target arity;
	however, this requires keeping track of an order of the vertices, independent of the planarization,
	so care is needed in practice).
	On a side note,
	\cite[Prop. 4.3.16]{WY18} is a 
	direct colored generalization
	of \cite[(7.12)]{Ha10},
	which is an abstraction of 
	\cite[Prop. 4.20]{Ha10}.
	In \cite[Prop. 5.20]{Pe16},
	the second author reproved this filtration result by identifying a free algebra extension
	as a colimit over a category $\mathcal{W}$
	(\cite[Def. 5.26]{Pe16}),
	then identifying a final subcategory of $\mathcal{W}$
	and filtering that subcategory.
\end{itemize}	


In practice, all approaches (LT),(PM),(CO)
run into technical issues 
that would require significant modifications 
when applied to Theorems 
\ref{MAINEXIST1 THM} and \ref{MAINEXIST2 THM},
as follows:
\begin{itemize}
	\item[(I)]
	First, for Theorem
	\ref{MAINEXIST1 THM},
	note that our free operad monad $\mathbb{F}$
	is, as below, the adjoint to forgetting to symmetric sequences.
	However, both approaches
	(PM) and (CO) work with the composite monad 
	$\widetilde{\mathbb{F}}$
	forgetting to just sequences (with no symmetric group actions).
	\begin{equation}
	\begin{tikzcd}[column sep =9em]
	\mathcal{V}^{\mathbb{N}_0}
	\ar[shift left=1.5]{r}
	{(X_n) \mapsto
		(\Sigma_n \cdot X_n)}
	&
	\mathcal{V}^{\Sigma} 
	\arrow[l, shift left=1.5] 
	\arrow[r, shift left=1.5,swap,"\mathbb{F}"']
	&
	\mathsf{Op}(\mathcal{V})
	\ar[shift left=1.5]{l}
	\end{tikzcd}
	\end{equation}
	However, setting $\mathcal{V}=\mathcal{V}^G$ above,
	the nature of graph subgroups $\Gamma \leq G \times \Sigma$ implies that
	one can only build the generating (trivial) cofibrations in 
	Theorem \ref{MAINEXIST1 THM} by using the monad $\mathbb{F}$,
	as $\widetilde{\mathbb{F}}$ can only build the generators for trivial graph subgroups $\Gamma \leq G$.
	\item[(II)] approach (LT) follows a 
	``build first, check latter'' philosophy, 
	where the underlying symmetric sequence of a free extension is first built directly using labeled trees, 
	with the two tasks of checking that the built sequence is
	an operad and a pushout in operads
	being handled a posteriori (and often left as an exercise to the reader). 
	But since explicitly describing all the compositions/restrictions/compatibilites
	of a genuine equivariant operad isn't easy,
	it seems hard to approach 
	Theorem \ref{MAINEXIST2 THM}
	via a ``build first, check latter'' philosophy
\end{itemize}

In practice, issue (II) seems the most prohibitive,
and in the early stages of this paper
we were planning to follow approach (CO),
with which we were more familiar 
(and had promising work arounds for issue (I)), 
as we were at first convinced that $G$-trees
could be assembled into a colored operad.
However, we eventually realized that this can not be the case, since defining genuine operads requires using diagonal maps.
This caused us to realize that genuine operads in the sense of this paper are instead algebras
over a \emph{colored} genuine operad $\mathsf{Op}_G$ of $G$-trees,
forcing us to drop approach (CO).

The approach we eventually followed in this paper is 
a hybrid of the philosophies in (LT),(PM),(CO).
As in (LT), we focus directly on the trees themselves throughout
(whereas in both (PM),(CO)
the trees are packaged as abstract operations)
but, as in (PM),(CO)
we follow a ``define first, then describe'' philosophy,
where we start with a free extension defined as an 
abstract object,
then work our way towards a description.

In practice, this means that while the labeled trees of 
\cite{BM03},\cite{Cav} are presented as an ``educated guess'' that is then checked to work out,
our approach provides a step by step procedure for 
building $\Omega^e_G$
without ``guessing'', 
much like \cite{BB17} provides a procedure for building
$\bold{T}^{T_{f,g}}$
and \cite{WY18,Ha10,Pe16}
provide procedures for building $\mathcal{O}_A$.

On a side note,
one advantage of having a procedural approach is that it is surprisingly easy to ``guess wrong''
when trying to build extension trees by hand,
mainly due to the roles of operadic units and of algebra units (i.e. stumps).
For example, 
in \cite{BP_HGOP},
which generalize Theorem \ref{MAINEXIST1 THM} to the colored case 
or, equivalently, generalizes 
\cite{Cav} to the equivariant case,
we adapt the techniques in this paper to describe general pushouts
in the category $\mathsf{Op}_{\mathfrak{C}}$ of operads with a fixed set of colors $\mathfrak C$.
Our answer (in \cite[App. A.5]{BP_HGOP}) involves left Kan extensions over a category
$\Omega^p_{\mathfrak{C}}$ of pushout trees 
that recovers (up to slight repackaging technicalities)
the description in 
\cite[App. B]{Cav}.
However, in \cite[App. C]{Cav}, 
which tackles the subject of free extensions, 
the ``categories of extension trees'' built therein
don't quite match $\Omega^e_G$.
More precisely, \cite[App. C]{Cav} forbids degeneracies,
and we believe that this leads the construction therein to
not handle the operadic unit correctly,
and thus not to define an operad (notably, that verification is left to the reader).





\bibliography{biblio}{}

\bibliographystyle{alpha}


\end{document} 




%%% Local Variables:
%%% mode: latex
%%% TeX-master: t
%%% End:
