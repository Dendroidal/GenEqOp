% LaTeX file for resume 
% This file uses the resume document class (res.cls)

\documentclass{article} 
% the margin option causes section titles to appear to the left of body text 
\textwidth=5.2in % increase textwidth to get smaller right margin
%\usepackage{helvetica} % uses helvetica postscript font (download helvetica.sty)
%\usepackage{newcent}   % uses new century schoolbook postscript font 
\usepackage{url}
\usepackage{hyperref}
\hypersetup{
  % colorlinks,
  final,
  pdftitle={Equivariant Dendroidal Segal Spaces},
  pdfauthor={Bonventre, P. and Pereira, L. A.},
  % pdfsubject={Your subject here},
  % pdfkeywords={keyword1, keyword2},
  linktoc=page
}

\usepackage{xr}
\externaldocument{GenEqOp}

\input{commands.tex}%

%-------- TIKZ -----------------------------------------
\usepackage{tikz}%
\usetikzlibrary{matrix,arrows,decorations.pathmorphing,
cd,patterns,calc}
\tikzset{%
  treenode/.style = {shape=rectangle, rounded corners,%
                     draw, align=center,%
                     top color=white, bottom color=blue!20},%
  root/.style     = {treenode, font=\Large, bottom color=red!30},%
  env/.style      = {treenode, font=\ttfamily\normalsize},%
  dummy/.style    = {circle,draw,inner sep=0pt,minimum size=2mm}%
}%

\usetikzlibrary[decorations.pathreplacing]
% \usetikzlibrary{external}\tikzexternalize
% \makeatletters
% \renewcommand{\todo}[2][]{\tikzexternaldisable\@todo[#1]{#2}\tikzexternalenable}

% \makeatother

\begin{document} 
 
\title{Edits to ``Genuine Equivariant Operads'' (v.2)
\\[12pt]} % the \\[12pt] adds a blank line after name
 
\author{Bonventre, P. and Pereira, L. A.}
 
\maketitle
 
The referee's preliminary report had two main sections:
high level comments and in line comments.



\section{High level}


\section{Changes before report}

\begin{itemize}
\item Updated Gutierrez-White reference

\item Fixed last square diagram in the proof of Lemma 6.69

\item clarified diagram (3.1)

\item clarified 2 sentences after Remark 3.13

\item clarified diagram in Example 3.18

\item Defn 3.19 and Example 3.20 became Defn 3.23 and Example 3.24, while 
Defn 3.21, Remark 3.22, Prop 3.23 became Defn 3.19, Remark 3.20, Prop 3.21

\item Fixed some typos in Lemma 2.21
\end{itemize}

\section{Direct response to comments}

\begin{itemize}
	\item[21.] we are unaware of any reference for Props. 
	\ref{FIBERKANMAP PROP} and \ref{GROTHSTAB PROP}.
	Certainly these results are elementary enough that they are probably already known (especially Prop. \ref{GROTHSTAB PROP}),
	but the results were new to us. 
	\item[24.] While these results are straightforward 
	and their proofs are generic, 
	we are unaware of any references for Propositions \ref{MONADADJ1 PROP}, \ref{MONADADJ PROP}, \ref{MONADICFUN PROP}. The likely references we are familiar with, such as 
	\cite{Bo94} or {\color{red} Rhiel} do not quite discuss the framework we need.
	As with Props. 
	\ref{FIBERKANMAP PROP} and \ref{GROTHSTAB PROP},
	these results were new to us.
	\item[46.] \eqref{PIIDEFDI2 EQ} does not need natural isomorphisms on the left, i.e. the square strictly commutes. The intuition is that since degeneracies merely duplicate one layer of vertex data, they do not change the way the vertex data is planarized. 
	\item[64.]
	In Corollary \ref{KANRED COR} the sentence immediately after ``$\widehat{\Omega}^e_G \hookrightarrow \Omega^e_G$ is
	$\mathsf{Ran}$-initial over $\Sigma_G$''
	is the definition of $\mathsf{Ran}$-initial,
	so it seems the comment was already addressed.
	In addition, Lemma \ref{MINUS_LAN_FINAL_LEMMA} where ``$\mathsf{Ran}$-initial'' also appears was altered to refer to Corollary \ref{KANRED COR}.
	\item[82.] the proof of Prop. \ref{POWERF PROP} is terse since the overall argument is technically identical to that in the proof of [22, Thm 1.2] (one just modifies the hypothesis to get different conclusions).
	We would be happy to add extra detail if deemed necessary. 
\end{itemize} 


\section{Minor and typos} 


\begin{itemize}
\item[1.] added accent to Guti\'{e}rrez's name throughout.

\item[14.] replaced ``simplicial operad'' with  ``$G$-simplicial operad''

\item[19.] added reference for the notion of ``Grothendieck fibration'' and reworded the definition of ``cartesian arrow''; fixed indicated grammar typo

\item[20.] fixed typos in Definition \ref{GROTHCONS DEF}

\item[21.] edited Prop. \ref{FIBERKANMAP PROP} to refer to the notation for a map of Grothendieck fibrations in \eqref{GROTHFIBMAP EQ}

\item[22.] fixed ``singleton''

\item[26.] the wrong reference seems to have been a compilation error

\item[29.] confirmed that the word ``predecessor'' (rather than ``sucessor'') is correct

\item[32.] fixed the $\tilde{G}$ notation to $\bar{G}$ and reworded the example to clarify that $G,\bar{G}$ are two unrelated groups

\item[38.] no change was requested

\item[41.] the wrong reference seems to have been a compilation error

\item[42.] the wrong reference seems to have been a compilation error

\item[43.] the wrong reference seems to have been a compilation error

\item[57.] fixed ``forgetful''

\item[65.] noted that the use of opposite categories turns $\mathsf{Ran}$ into $\mathsf{Lan}$

\item[70.] replaced ``less generating'' with ``fewer generating''

\item[77.] fixed $G$ to be $\bar{G}$

\item[80.] reworded ``for all'' as ``for every'' in Prop. \ref{POWERF PROP}

\item[82.] rephrased ``retracts can be ignored''
as ``retracts preserve weak equivalences''

\item[83.] fixed ``trully'' to ``truly''

\item[93.] added accent to Gutierrez's name throughout.

\item[97.] fixed wrong way quotation mark

\item[98.] rephrased the proof of Lemma \ref{OBJGENREL LEMMA}
to avoid $\in$ symbols in opposite directions

\item[100.] clarified the reference to Prop. \ref{FIBERKANMAP PROP} by specifying the relevant undercategories

\end{itemize}








\bibliography{biblio}{}

\bibliographystyle{alpha}


\end{document} 




%%% Local Variables:
%%% mode: latex
%%% TeX-master: t
%%% End:
