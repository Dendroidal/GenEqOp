\documentclass[a4paper,10pt]{article}%


\usepackage[hidelinks]{hyperref}
\hypersetup{
%  colorlinks,
  linktoc=page
}


\input{commands.tex}%

\author{Peter Bonventre, Lu\'is A. Pereira}%
\title{Genuine equivariant operads}%

\usepackage{showkeys}

\usepackage{stmaryrd}

\usepackage{geometry}

\usepackage{tikz}%
\tikzset{%
  treenode/.style = {shape=rectangle, rounded corners,%
                     draw, align=center,%
                     top color=white, bottom color=blue!20},%
  root/.style     = {treenode, font=\Large, bottom color=red!30},%
  env/.style      = {treenode, font=\ttfamily\normalsize},%
  dummy/.style    = {circle,draw,inner sep=0pt,minimum size=2mm}%
}%

\usetikzlibrary[decorations.pathreplacing]
%\usetikzlibrary{external}\tikzexternalize

\begin{document}	\maketitle%

Given 
$\mathsf{Fin} \wr \mathcal{V} \to \mathcal{V}$ with suitable isomorphisms $\alpha$, we define the associators by
\[
\begin{tikzcd}
	(A \cdot B) \cdot (C) \ar{r}{\alpha} &
	A \cdot B \cdot C &
	(A) \cdot (B \cdot C) \ar{l}[swap]{\alpha}
\end{tikzcd}
\]
and the unit morphisms by
\[
\begin{tikzcd}
	(A) \cdot () \ar{r}{\alpha} &
	A &
	() \cdot (A) \ar{r}{\alpha} &
	A
\end{tikzcd}
\]

The associativity ``pentagon'' axiom follows from
\[
\begin{tikzcd}[column sep=5pt]
	&
	((A \cdot B)) \cdot ((C) \cdot (D)) 
	\ar[equal]{r}{\mathsf{F} \wr \alpha} \ar{ld}[swap]{\alpha} &
	(A \cdot B) \cdot (C \cdot D) \ar{dd} &
	((A) \cdot (B)) \cdot ((C \cdot D)) 
	\ar[equal]{l}[swap]{\mathsf{F} \wr \alpha} \ar{rd}{\alpha}
\\
	(A \cdot B) \cdot (C) \cdot (D) \ar{rrd} & 
	&
	&
	&
	(A) \cdot (B) \cdot (C \cdot D)\ar{lld}
\\
	((A \cdot B) \cdot (C)) \cdot ((D)) \ar{u}{\alpha}
	\ar{d}[swap]{\mathsf{F} \wr \alpha}& 
	&
	A \cdot B \cdot C \cdot D
	&
	&
	((A)) \cdot ((B) \cdot (C \cdot D)) \ar{u}[swap]{\alpha} \ar{d}{\mathsf{F} \wr \alpha}
\\
	(A \cdot B \cdot C) \cdot (D) \ar{rru} &
	&
	&
	&
	(A) \cdot (B \cdot C \cdot D) \ar{llu}
\\
	&
	((A) \cdot (B \cdot C))\cdot ((D)) \ar{r}[swap]{\alpha} \ar{lu}{\mathsf{F} \wr \alpha} &
	(A) \cdot (B \cdot C) \cdot (D) \ar{uu} &
	((A)) \cdot ((B \cdot C) \cdot (D)) \ar{l}{\alpha} 
	\ar{ru}[swap]{\mathsf{F} \wr \alpha}
\end{tikzcd}
\]

The identity axiom is

\[
\begin{tikzcd}
	((A) \cdot ()) \cdot ((B)) \ar{r}{\alpha} \ar{d}[swap]{\mathsf{Fin} \wr \alpha} &
	(A) \cdot () \cdot (B) \ar{d} &
	((A)) \cdot (() \cdot (B)) \ar{l}[swap]{\alpha}
	\ar{d}{\mathsf{Fin} \wr \alpha}
\\
	(A) \cdot (B) \ar[equal]{r}&
	A \cdot B &
	(A) \cdot (B) \ar[equal]{l}
\end{tikzcd}
\]

The symmetry morphism is defined in the obvious way as 
$\otimes (\tau)$ for $\tau$ the 
isomorphism
$(A,B) \simeq (B,A)$ in $\Sigma_2 \wr \mathcal{V}$.

The inverse law is then simply automatic and the coherence of symmetry with unit is naturality of $\alpha$, as in the diagram
\[
\begin{tikzcd}
	(A) \cdot () \ar{r}{\tau} \ar{d}{\alpha} &
	() \cdot (A) \ar{d}{\alpha}
\\
	A \ar{r}[swap]{\tau} &
	A
\end{tikzcd}
\]

Lastly, associativity coherence is then ensured by the diagram (where the center triangle commutes since it is in the image of $\Sigma_3 \wr \mathcal{V}$)
\[
\begin{tikzcd}[column sep = 5pt]
	&
	&
	(A)\cdot (B \cdot C) \ar{rr}{\tau} \ar{dl}[swap]{\alpha}&
	&
	(A) \cdot (C \cdot B)\ar{dr}{\alpha}
\\
	&
	A \cdot B  \cdot C \ar{rrrr}{\tau} \ar{ddrr}[swap]{\tau}&
	&
	& 
	&
	A \cdot C \cdot B \ar{ddll}{\tau}
\\
	(A \cdot B) \cdot (C) \ar{ru}{\alpha} \ar{rd}[swap]{\tau}&
	&
	&
	&
	&
	&
	(A \cdot C)  \cdot (B) \ar{lu}[swap]{\alpha} \ar{dl}{\tau}
\\
	&
	(B\cdot A) \cdot (C) \ar{rr}[swap]{\alpha} &&
	B \cdot A \cdot C &&
	(B) \cdot (A \cdot C)	\ar{ll}{\alpha}
\end{tikzcd}
\]




\bibliography{biblio}{}



\bibliographystyle{abbrv}



\end{document}