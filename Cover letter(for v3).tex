% LaTeX file for resume 
% This file uses the resume document class (res.cls)

\documentclass{article} 
% the margin option causes section titles to appear to the left of body text 
\textwidth=5.2in % increase textwidth to get smaller right margin
%\usepackage{helvetica} % uses helvetica postscript font (download helvetica.sty)
%\usepackage{newcent}   % uses new century schoolbook postscript font 
\usepackage{url}
\usepackage{hyperref}
\hypersetup{
  % colorlinks,
  final,
  pdftitle={Genuine Equivariant Operads},
  pdfauthor={Bonventre, P. and Pereira, L. A.},
  % pdfsubject={Your subject here},
  % pdfkeywords={keyword1, keyword2},
  linktoc=page
}

\usepackage{xr}
\externaldocument{GenEqOp}

\input{commands.tex}%

%-------- TIKZ -----------------------------------------
\usepackage{tikz}%
\usetikzlibrary{matrix,arrows,decorations.pathmorphing,
cd,patterns,calc}
\tikzset{%
  treenode/.style = {shape=rectangle, rounded corners,%
                     draw, align=center,%
                     top color=white, bottom color=blue!20},%
  root/.style     = {treenode, font=\Large, bottom color=red!30},%
  env/.style      = {treenode, font=\ttfamily\normalsize},%
  dummy/.style    = {circle,draw,inner sep=0pt,minimum size=2mm}%
}%

\usetikzlibrary[decorations.pathreplacing]
% \usetikzlibrary{external}\tikzexternalize
% \makeatletters
% \renewcommand{\todo}[2][]{\tikzexternaldisable\@todo[#1]{#2}\tikzexternalenable}

% \makeatother

\begin{document} 
 
\title{Edits to ``Genuine Equivariant Operads'' (v3)
\\[12pt]} % the \\[12pt] adds a blank line after name
 
\author{Bonventre, P. and Pereira, L. A.}
 
\maketitle

The referee's report requested only a number of small changes along with a careful reread of Appendix B),
both of which were addressed. In the following sections we list the changes made to v3.

\section{Responses to the edits}

The numbering for the changes in this section refers to the numbering in the ``Responses to the Edits Document'' section of the report.

\begin{itemize}
\item[(71)] The reference pointed to does provide the precise analogue needed.
        We have thus updated our references for the definition of semi-model structure,
        as well as simplified our narrative in the proof of Theorems \ref{MAINEXIST1 THM} and \ref{MAINEXIST2 THM}.
\end{itemize}

\section{Changed in response to ``Line-by-line editorial remarks''}

The numbering for the changes in this section refers to the numbering in the ``Line-by-line editorial remarks'' section of the report.

\begin{itemize}

\item[(1)] As requested, in Example \ref{GENALG EX} we rewrote ``...the sets $\pi_0(X)$, $\pi_0(X^G)$ admit...'' as ``...the sets $\pi_0(X)$ and $\pi_0(X^G)$ admit...''.

\item[(2)] Before \eqref{BARLEVELS EQ} we
fixed the $n$ that was appearing in $B_{\bullet} = 
B_{\bullet}(N,N,\upsilon \mathcal{P})
= N^{\bullet+1} \upsilon \mathcal{P}$
to be a $\bullet$.

item[(3)] Before \eqref{BARLEVELS EQ} we added a further reminder of where $\upsilon$ is defined and pointed to a relevant remark. We did not add $\upsilon \mathcal{P}$ to the glossary, since this feels somewhat redundant, as there is already an entry for $\upsilon$ itself.


\item[(4)] In Remark \ref{WHYALT REM} the typo ``reorded'' was fixed.

\item[(5)]
In the text after Remark \ref{WRONGSTRAT REM} we fixed
``with an analogue construction'' to be 
``with an analogous construction''.

\item[(6)] added graph subgroups $\Gamma_X$ to the index. {\color{red} to Peter: somewhat strangely, the glossary isn't producing the correct spacing for the new graph subgroups item, as the ``graph subgroups'' and ``Grothendieck fibrations'' categories have no space between them...}

        
\end{itemize}









\bibliography{biblio}{}

\bibliographystyle{alpha}%alpha


\end{document} 




%%% Local Variables:
%%% mode: latex
%%% TeX-master: t
%%% End:
